\documentclass[a4paper]{article}

\usepackage[T1]{fontenc}
\usepackage[utf8]{inputenc}
\usepackage[italian]{babel}
\usepackage{epigraph}

\title{Relazione Fake News}
\author{Gabriele Bellavia}

\begin{document}
\maketitle

\section*{Introduzione.}
\epigraph{La sapienza è figliola della sperienzia.}%
{Leonardo da Vinci, \\ Codice Forster III}
Nell'informazione scientifica sono fondamentali le informazioni corrette, alle quali si applicano giuste regole logiche, ovvero deduzione, induzione e abduzione, che permettono di arrivare a una conclusione accettabile da premesse o ipotesi valide e quindi creare conoscenza.
Tuttavia, tramite dei meccanismi dovuti a errori cognitivi o bias, cattive informazioni o interpretazione errata di esse, le conclusioni a cui si arriva sono sbagliate e creano una conoscenza distorta.\\
I protagonisti della divulgazione devono prendere delle precauzioni per far sì che venga creata una giusta conoscenza alla quale punta l'informazione scientifica.

\section*{I protagonisti e le caratteristiche della \mbox{divulgazione}.}
I protagonisti della divulgazione di un'informazione scientifica sono
i divulgatori e il pubblico.
I divulgatori sono i responsabili della creazione dell'informazione, che è suscettibile a diversi meccanismi, ovvero la disinformazione, la misinformazione e la malinformazione.
La disinformazione consiste nella creazione e divulgazione volontaria di una notizia falsa, a differenza della misinformazione che è la divulgazione
involontaria di una notizia falsa che si crede vera. La
malinformazione è la divulgazione di una notizia vera, che tuttavia
viene condivisa per causare danno in un particolare contesto.
Esiste un'associazione che controlla e cerca di smentire informazioni rilasciate da enti pseudoscientifici poco affidabili che si chiama "Comitato Italiano per il controllo delle affermazioni sulle
pseudoscienze" (CICAP).



\section{La presentazione della notizia.}
La presentazione delle notizie è cambiata nel corso del tempo a causa
dello sviluppo dei mezzi d'informazione.
Oggi si ha una diffusione maggiore tramite i social, di conseguenza le notizie online devono avere un titolo accattivante per attirare l'attenzione di più
persone, sia a causa di un'enorme quantità di notizie, ma anche per il meccanismo del "Pay per click", ovvero della remuneratività della notizia in base al numero di persone che "Cliccano" per leggerla.\\
La diffusione online comporta anche una minore revisione del materiale e tra le notizie diffuse ci sono molte "fake news" con contenuti poco curati che possono anche non avere alcuna affidabilità, ma sono visualizzate da molte persone.


\subsection{Artifici scientifici.}
La raccolta e manipolazione dei dati è un processo che dà validità all'informazione che viene proposta dalla notizia.
La raccolta dei dati è un processo cruciale per ottenere risultati che confermino un'ipotesi, perciò la scelta di un campione di riferimento rappresentativo e l'analisi dei casi rappresentati sono fondamentali per definire i rapporti di causa-effetto dei fenomeni analizzati. \\
Gli artifici scientifici sono estrapolazioni errate dei dati e ne esistono di diversi tipi.
I ragionamenti induttivi possono essere spesso causa di fallacie, come viene riassunto bene da un esempio di Karl Popper ovvero del "Tacchino induttivista". A un tacchino viene data ogni giorno una razione di cibo, perciò il tacchino ogni giorno aspetta la sua razione ed è sicuro che la riceverà. Tuttavia, un giorno, il tacchino non riceve nulla e viene servito a tavola, proprio il giorno della "Festa del ringraziamento" (Thanksgiving day).

È frequente anche una scelta errata del campione di riferimento, in particolare se dà informazioni coerenti con l'ipotesi, il cosiddetto "p-hacking", metodo chiamato anche "raccogliere ciliegie" o del "cecchino texano", ovvero scegliere i dati più "ottimisti" della ricerca per confermare la propria ipotesi.
Inoltre, sono frequenti le cosiddette "correlazioni spurie", ovvero l'apparente relazione di causa-effetto tra due fenomeni completamente indipendenti.

I risultati dell'estrapolazione dei dati sono significativi e dei risultati corretti sono
la parte più importante per verificare la verità di una ricerca.

\section{Il pubblico e le interferenze.}
La divulgazione scientifica ha un ruolo educativo e ispirazionale per il pubblico, tuttavia ciò che interferisce, anche nei casi di validità della notizia scientifica, non permette un cambiamento interiore nel pubblico.
La diffusione di notizie false ha una radice culturale e antropologica perché il pubblico ha delle preferenze in merito ad alcuni temi e può essere conservativo nei riguardi della sua attuale conoscenza.
A volte l'evidenza scientifica non basta, è difficile influenzare sistemi di valori, convinzioni e ideali radicati nel tempo, ma è necessaria un'educazione per provocare un cambiamento culturale e antropologico nella popolazione che si riflette in manifestazioni esteriori e diventano abitudini.

\subsection{L'euristica dell'informazione.}
Gli errori cognitivi sono degli espedienti per trarre sollievo da una notizia; si accettano notizie che confermano le nostre convinzioni, si trovano le soluzioni più semplici o i "Capri espiatori" più prossimi da evitare per non subire un danno.
Un esempio è l'avversione alle sostanze chimiche, la creazione di prodotti "Chemical-free" senza sostanze chimiche, perché esse vengono considerate dannose per la salute, tuttavia ogni sostanza, anche naturale, è "Chimica" in senso stretto.
Da ciò consegue lo scherzo della "DHMO" ovvero il monossido di diidrogeno, che viene presentata alla vittima dello scherzo come una sostanza pericolosissima, un solvente molto potente, ma è solamente l'acqua con un nome scientifico.

In occasione della  "European Breast Cancer Conference" (EBCC-3) del 2002, nella quale il tema centrale era il carcinoma mammario, venne presentata una ricerca effettuata su 103.000 donne norvegesi e svedesi.
In questa ricerca si dava come dato un rischio basale del 7\% per questa patologia, ma la notizia che creò scalpore fu l'aumento del 26\% del rischio di cancro alla mammella per le donne che assumevano contraccettivi orali.
La notizia creò scalpore perché a prima vista l'aumento sembrava davvero enorme e creò subito preoccupazione, ma in termini di rischio relativo (RR), ovvero l'aumento del rischio per il gruppo di donne che assumevano contraccettivi orali, il RR è uguale a 1.26, perciò la nuova percentuale di rischio del gruppo esposto è 8.82\%, quindi quasi due donne in più su cento si ammaleranno nel gruppo che assume contraccettivi orali.\\ Le notizie possono avere funzione di propaganda causando preoccupazione a un'analisi poco attenta.

Perciò per evitare una manipolazione da parte delle notizie, ogni membro del pubblico dovrebbe controllare le fonti, confrontarle e allenare la propria capacità
critica per discernere le notizie false o estrapolare la verità sottostante che si cerca di comunicare. Infine, non si deve odiare ciò che si disconosce, proprio perché è la maggior parte del mondo quella che è ancora a noi oscura.
\end{document}
