\documentclass[12pt, a4paper]{article}


\usepackage[T1]{fontenc}
\usepackage[utf8]{inputenc}
\usepackage[italian]{babel}
\usepackage{graphicx}
\usepackage{hyperref}

\title{Il saggio di Bradford in poche parole.}
\author{Gabriele Maria Bellavia}
\date{\today}

\begin{document}

\maketitle

Tutto il materiale presentato si basa su ciò che ho imparato nella mia carriera universitaria.\\
I dati riportati nella sezione `Protocollo' sono stati ricopiati da un foglio intitolato `Protocollo Esercitazione' di cui sono entrato in possesso durante il periodo in cui ho svolto l'attività di tirocinio presso il laboratorio di Genomica dell'edificio 16 dell'Università degli Studi di Palermo.


Il `Saggio proteico di Bradford' o `Bradford Protein assay' è un metodo di laboratorio, descritto per la prima volta dal Dr.{}Marion Bradford nel 1976, per stabilire la concentrazione totale di un campione proteico in soluzione confrontandolo con una serie di soluzioni di una proteina (BSA) a concentrazione nota.


\section{Reagenti}
I due reagenti fondamentali previsti da questa tecnica sono il \mbox{`Blue di Coomassie'} e la `BSA' - l'albumina di siero bovino.
Descriverò brevemente solo la preparazione del primo.
Altri reagenti sono importanti, ma non sono propri di questa tecnica, come il tampone M-RIPA (buffer).


\textbf{N.B.}
Esistono due versioni della tecnica: \emph{macro} e \emph{micro}.
La differenza è il potere risolutivo del metodo, infatti la seconda è adatta a quantificare concentrazioni molto basse di proteine.
Di conseguenza, nella prima si utilizza una soluzione standard di BSA a concentrazione 4mg/mL, inoltre si aggiungono 4mL di reattivo di Bradford alla soluzione da misurare; nella \emph{micro} la BSA è $0,4$mg/mL e si aggiunge solo 1mL di reattivo.

\subsection*{Blue di Coomassie}
\includegraphics[width=200px]{coomassie}


\`E il vero e proprio `Reattivo di Bradford', ha il nome `Coomassie Brilliant Blue G-250'.
Forma un complesso con i residui aminoacidici basici e aromatici (specialmente di lisina e arginina) presenti in una proteina.
Esiste in tre forme:
\begin{itemize}
\item cationica, (picco di assorbanza $\lambda=465$nm, colore rosso--marrone)
\item neutra, (picco di assorbanza $\lambda=650$nm, colore verde)
\item anionica, (picco di assorbanza $\lambda=595$nm, colore blu)
\end{itemize}

Il complesso proteina--colorante causa uno shift del picco di assorbanza da 465nm a 595nm, favorendo così la forma deprotonata del colorante e sviluppando il colore blu.



In breve (giusto per saperlo, non è una descrizione accurata), 1L di colorante si prepara: \\
Aggiungendo 100mg di Coomassie Brilliant Blue G-250 in 50mL di Etanolo. \\
Successivamente si aggiungono 100mL di acido fosforico all' 85\% e si porta a volume. \\
Si filtra e si conserva a 4°C.

\subsection*{BSA - Albumina di siero bovino}
\`E una proteina generica, ha una singola catena polipeptidica costituita da circa 583 amminoacidi, priva di carboidrati.

\begin{center}
\includegraphics[width=250pt]{albumina}
\begin{figure}[!h]
\cite{pianeta-chimica}
\end{figure}
\end{center}

\section{Protocollo}
Si procede con la preparazione di 8 soluzioni a concentrazione crescente e una soluzione priva di BSA detta `BLANK', a cui corrisponderà un segnale di base, per eliminare il rumore di fondo dovuto all'errore intriseco dello strumento (spettrofotometro).

\begin{table}[!h]
\begin{tabular}{c|c|c|c}
CONCENTRAZIONE [mg/mL] & BSA [µL] & TAMPONE [µL] & H2O [µL] \\
\hline
0.1 & 2.5 & 10 & 87.5 \\
0.2 & 5 & 10 & 85 \\
0.4 & 10 & 10 & 80 \\
0.6 & 15 & 10 & 75 \\
0.8 & 20 & 10 & 70 \\
1 & 25 & 10 & 65 \\
1.2 & 30 & 10 & 60 \\
1.4 & 35 & 10 & 55 \\
BLANK & --- & 10 & 90 \\
\end{tabular}
\caption{Soluzioni a concentrazione crescente.}
\end{table}

\begin{center}
\includegraphics[width=400pt]{regressionebradford}
\begin{figure}[!h] 
\cite{bsa-paper}
\end{figure}
\end{center}

Dopo la costruzione della curva standard, si possono misurare i campioni a concentrazione sconosciuta.
La curva serve (il metodo numerico si chiama `Regressione lineare')  a creare una corrispondenza tra un valore misurato di assorbanza del nostro campione, in ordinata, e un valore di concentrazione, in ascissa.
Si aggiunge 1mL (nella versione Macro si aggiungono 4mL) di reattivo di Bradford, dopo 5--10 minuti si pipettano 2mL in una cuvette e si legge l'assorbanza allo spettrofotometro a 595nm.


\section{Contesto di laboratorio}
Questa descrizione non ha finora collocato il saggio in una posizione precisa all'interno di un esperimento.
Per fare un esempio, basandomi sulla mia esperienza di laboratorio, un suo utilizzo sarebbe previsto in questa sequenza:

\begin{enumerate}
\item Prelievo di cellule da una coltura (e.g. MDA)
\item Lisi delle cellule con un tampone (ecco, l' M-RIPA viene introdotto in questo punto, è nella soluzione di lisi, nel caso delle suddette cellule)
\item Centrifugazione e recupero del sovranatante. Se necessario, effettuare anche una concentrazione del campione (esiste una centrifuga particolare)
\item misurazione della concentrazione proteica. (BRADFORD!!!)
\item PCR, western blot e altre analisi \ldots
\end{enumerate}

Ovviamente la preparazione di una curva standard deve tener conto del solvente in cui sono disciolte le proteine del campione a concentrazione sconosciuta.
Infatti, nell'esempio considero un'estrazione proteica per uno studio di proteomica da cellule MDA.
La BSA dovrà essere disciolta in un tampone contenente M-RIPA, ciò permette di quantificare meglio il campione.

\begin{thebibliography}{2}
\bibitem{pianeta-chimica} \url{https://www.pianetachimica.it/mol_mese/mol_mese_2003/01_Albumina/Albumina_1_ita.html}
\bibitem{bsa-paper} \url{https://www.researchgate.net/profile/Ibrahim-Naqid/publication/305287157/}
\end{thebibliography}


\end{document}
